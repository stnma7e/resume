\documentclass{tccv}
\usepackage[english]{babel}
\usepackage{hanging}

\begin{document}

\Header{Sam Delmerico}
    {(404) 786-3104}
    {Atlanta, GA}
    {}
    {svdelmerico@gmail.com}
    {github.com/stnma7e}
    {linkedin.com/in/sam-delmerico}

\begin{minipage}[t]{0.35\textwidth}
    \section{Education}
    \begin{tabular}{rl}
        2019 & \textbf{Chemistry, B.S.} \\
             & \emph{Georgia Tech} \\
             & \emph{Highest Distinction} \\
             & GPA: 3.71 \\
             & Dean's List \\
        
        2017 & \textbf{Chemistry} \\
        to   & \emph{Georgia State University} \\
        2016 & GPA: 4.11 \\
             & Honors College \\
             & President's List
    \end{tabular}

    \section{Computer Skills}
    \textbf{\textsc{Languages}}
    
    \textsc{Proficient:} Python - Rust - Javascript
    
    \textsc{Familiar:}
    {
        C/C++ - Java - C\# - Haskell - Go
    } \\
    
    \textbf{\textsc{OS}}:
        Linux, Windows
   
    \textbf{\textsc{Tools}}:
        Git, Jupyter, \LaTeX{} \\
    
    \textbf{Courses} \\
        Intro to CS \newline
        Intro to OOP \newline
        Intro to Data Structures and Algorithms \newline
        Linear Algebra I \& II \newline
        Quantum Computing and Information Theory
        
    \section{Awards and Presentations}
    
    \textbf{President's Undergraduate \\ Research Award (PURA)} \\
    \emph{Summer 2018}, Georgia Tech. I was selected to receive a stipend to conduct summer research with the McDaniel lab. \\
    
    \textbf{Georgia Tech Energy Club} \\
    \emph{Fall 2018}, I was invited to talk about computational research in supercapacitor design. \\
    
    \textbf{SERMACS 2018} \\
    \emph{Fall 2018}, Poster presentation on my work computationally modelling supercapacitors.
\end{minipage}
\begin{minipage}[t]{0.3\textwidth}
\end{minipage}
\begin{minipage}[t]{0.65\textwidth}
\section{Work experience}

\begin{eventlist}

\item{Jun 2019 -- Present}
     {Kabbage, Inc.}
     {Software Engineering Intern}
    
    I worked directly with an engineering team using an Agile methodology to pick up tickets and help them finish their sprints. My responsibilities were split between the C\#/.NET microservice-based backend API, and the frontend Angular web app. On the frontend I worked on bug fixes and user-facing feature implementation with idiomatic Angular tooling including functional reactive programming with RxJS. Both sides of development gave me experience with the dependency injection pattern and asynchronous programming patterns. Our deployment was entirely container based, giving me a deeper exposure to Docker-based workflows and some exposure to Kubernetes deployment with Terraform.

\item{Sept 2017 -- Present}
     {McDaniel Lab, Georgia Tech}
     {Undergraduate Research Assistant}

    I developed computational simulations of chemical systems. The simulations are written using a Python library and run on local Linux machines and on a computing cluster hosted by the school. Algorithms and mathematical analyses of the collected data are scripted in Python (often via Jupyter notebooks) making use of parsing libraries for extraction of relevant data.

\item{Jan 2017 -- Aug 2017}
     {Ivanov Lab, Georgia State University}
     {Undergraduate Research Assistant}

    I helped script data analysis programs for the Ivanov group, a computational biophysical chemistry lab.

\item{May 2017}
     {OnSolve, Vietnam}
     {Software Engineering Intern}

    While visiting my family in Vietnam for the summer, I set up an automatic product deployment pipeline by installing source control hooks to build the product with a cloud Jenkins instance and deploy production Docker containers to Amazon AWS.

\item{Aug 2016 -- Apr 2017}
     {Georgia State University}
     {Teaching Assistant}

    I assisted in three general chemistry labs during my first two semesters of college.

\section{Projects}

\href{https://stnma7e.github.io/schedulater}{\emph{schedulater}}, generates a list of valid course schedules for Georgia Tech classes from an input of desired options (written in Javascript/Elm; see Github). 
\\ \\
\emph{betuol}, a text-based game engine (written in Go; see Github)

\end{eventlist}
\end{minipage}

\end{document}
