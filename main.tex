\documentclass{tccv}
\usepackage[english]{babel}
\usepackage{hanging}
\usepackage{tabulary}

\begin{document}

\Header{Sam Delmerico}
    {(404) 786-3104}
    {Atlanta, GA}
    {}
    {svdelmerico@gmail.com}
    {github.com/stnma7e}
    {linkedin.com/in/sam-delmerico}

\small

\begin{minipage}[t]{0.35\textwidth}
    \section{Education}
    
    {\raggedright
        I am currently pursuing my undergraduate degree in Computer Science at Georgia Tech.
        \\
    }
    
    \vspace{10pt}
    
    \begin{tabular}{rl}
        2019 & \textbf{Chemistry, B.S.} \\
             & \emph{Georgia Tech} \\
             & \emph{Highest Honors Graduate} \\
             & GPA: 3.71 \\
             & Dean's List \\
 
        \\
        
        2017 & \textbf{Chemistry} \\
             & \emph{Georgia State University} \\
             & GPA: 4.11 \\
             & Honors College \\
             & President's List \\
             & \textbf{\emph{Transferred to GaTech}}
    \end{tabular}

    \section{Computer Skills}
    \textbf{\textsc{Languages}}
  
    \begin{tabulary}{\linewidth}{lL}
        \textsc{Proficient:} & Python - Rust - Javascript/Typescript \\
        \textsc{Familiar:} &  C/C++ - Java - C\# - Haskell - Fortran
    \end{tabulary}
    \\\\
    
    \textbf{\textsc{OS}}:
        Linux, Windows
   
    \textbf{\textsc{Tools}}:
        Git, Jupyter, \LaTeX{} \\
    
    \textbf{CS Courses} \\
        Intro to OOP \\
        Intro to Data Struct. and Algorithms \\
        Design and Analysis of Algorithms \\
        Quantum Comp. and Info. Theory \\
 
    \textbf{Math Courses} \\
        Applied Combinatorics \\
        Discrete Math \\
        Linear Algebra I \& II
        
    \section{Awards and Presentations}
    
    \textbf{President's Undergraduate \\ Research Award (PURA)} \\
    \emph{Summer 2018}, Georgia Tech. I was selected to receive a stipend to conduct summer research with the McDaniel lab. 
    
    \vspace{10pt}
    
    \textbf{Georgia Tech Energy Club} \\
    \emph{Fall 2018}, I was invited to talk about computational research in supercapacitor design. \\
    
    \textbf{SERMACS 2018} \\
    \emph{Fall 2018}, Poster presentation on my work computationally modelling supercapacitors.
\end{minipage}
\begin{minipage}[t]{0.3\textwidth}
\end{minipage}
\begin{minipage}[t]{0.65\textwidth}
\section{Work experience}

\begin{eventlist}

\item{Jun 2019 -- Aug 2019}
     {Kabbage, Inc.}
     {Software Engineering Intern}
     
    I worked directly with an engineering team using an Agile methodology to pick up tickets and help finish sprints. My responsibilities were split between the C\#/.NET microservice-based backend API, and the frontend Angular web app. On the frontend I worked on bug fixes and user-facing feature implementation with idiomatic Angular tooling including functional reactive programming with RxJS. Both sides of development gave me experience with the dependency injection pattern and asynchronous programming patterns. Our deployment was entirely container based, giving me a deeper exposure to Docker-based workflows and some exposure to Kubernetes deployment with Terraform.

\item{Sept 2017 -- Present}
     {McDaniel Lab, Georgia Tech}
     {Undergraduate Research Assistant}

    I worked on a project to find optimal strategies for supercapacitor design. This project involved programming a simulation using Python bindings to a C++/CUDA backend and running on local and remote Linux computing clusters. To analyze the results, other Python libraries were used to program analysis algorithms and extract relevant data (often via Jupyter notebooks).
    \\\\
    Aside from this, I work on a molecular dynamics code in Fortran to which I added a Monte Carlo algorithm for determining equilibrium system pressure and the capability to augment certain functions with neural network input.
    
\item{Jan 2017 -- Aug 2017}
     {Ivanov Lab, Georgia State University}
     {Undergraduate Research Assistant}

    In my first introduction to research, I setup a pipeline for running chemistry simulations on our GPU cluster, and I helped script data analysis programs for other memebers of the team. Notably, we used a PCA analysis to uncover physically relevant motions of a protein. 

\item{May 2017}
     {OnSolve, Vietnam}
     {Software Engineering Intern}

    While visiting my family in Vietnam for the summer, I set up an automatic product deployment pipeline for by installing source control hooks to build the product with a cloud-hosted Jenkins instance and deploy production Docker containers to Amazon AWS.

\item{Aug 2016 -- Apr 2017}
     {Georgia State University}
     {Teaching Assistant}

    I assisted in three general chemistry labs during my first two semesters of college.

\section{Personal Projects}

\href{https://stnma7e.github.io/schedulater}{\emph{schedulater}}, generates a list of valid course schedules for Georgia Tech classes from an input of desired options; essentially solving the knapsack problem (written in Javascript/Elm; see Github). 
\\ \\
\emph{betuol}, a text-based game engine (written in Go; see Github)

\end{eventlist}
\end{minipage}

\end{document}
